\documentclass{article}
\pagestyle{myheadings}

\usepackage[utf8]{inputenc}
\usepackage{apacite}

\begin{document}
\begin{titlepage}
\begin{center}
\LARGE{\textbf{The Realtime Assesment of Mental Workload by Means of Multiple Bio-Signals}}\\
\vspace*{2\baselineskip}
\Large{\textbf{Masterthesis Report}}\\
Methodology and Statistics for the Behavioural, Biomedical and Social Sciences\\
\vspace*{1\baselineskip}
Utrecht University\\
\vspace*{8\baselineskip}
{Bart-Jan Boverhof, 6000142}\\
\vspace*{1\baselineskip}
{\textbf{Thesis Supervisor}}\\
Prof.dr.ir. B.P. Veldkamp\\
\vspace*{1\baselineskip}
{\textbf{Date}}\\
October 14, 2020\\
\vspace*{1\baselineskip}
{\textbf{Word count}}\\
747\\
\vspace*{1\baselineskip}
\end{center}
\end{titlepage}

\section{Introduction}
The topic of mental workload (MWL) has received widespread attention across a variety of different fields, amongst others the field of ergonomics \cite{young2015state}, human factors \cite{pretorius2007development} and neurosciences \cite{shuggi2017mental}. A simple definition of MWL is the demand placed upon humans whilst conducting a specific task. As pointed out by de Waard (1996), such a definition is too shallow, for it defines workload solely externally. It is of importance to acknowledge that MWL is person-specific, for the amount of experienced MWL ushered by a given task differs across people \cite{de1996measurement}.

A commonly employed measure of workload is the NASA-Task Load Index questionnaire, operationalizing workload in clusters of six different dimensions \cite{hart2006nasa}. A potential problem arousing with such measurements, is that they are conducted post-experiment, posing the risk of jeopardizing the quality of the data by introducing subjectivity. An example of such a danger resides in the observer bias, proving that actors participating in an experiment tend to over-exaggerate the treatment effect when having to report it themselves post-experiment \cite{mahtani2018catalogue}. An approach inherently less prone towards such biases is the employment of bio-signals during an experiment.

The main objective of the current research is to provide a flexible framework with which researchers can measure MWL using the specific bio-signals (hereafter modalities) that are most feasible within their context.  The latter framework will aim to classify MWL in real-time, i.e. whilst the experiment takes place. Ultimately, this line of research pursues the ability to conduct a dynamic experiment, the development of which can be altered by in real-time. The following research question is delineated: \\ 
Which type of framework is most feasible for analyzing MWL in real-time by means of various bio-signals?

\newpage
\section{Data and Methods}
In order to answer the aforementioned question, data will be collected utilizing three different modalities, surrounding which the framework will be build. In order to attempt this in real-time, a technique called Lab Streaming Layer (LSL) is utilized during the data collection process. This enables data streams from all modalities to be properly synchronized, whilst satisfying the principle of modularity (i.e. the possibility to add various modalities) \cite{kothe2018lab}. More elaborate explorations of the appropriate architecture of this real-time framework are currently being endeavoured, but fall outside the scope (and word limit) of this proposal. The framework will be created in Python, by utilizing the machine learning toolbox Pytorch \cite{paszke2017automatic}

\subsection{Data}
The experimental setting for data collection is the spaceship bridge simulator videogame Empty Epsilon, in which respondents are required to carry out tasks on a virtual spaceship bridge \cite{daid2016empty}. This experiment is instituted by the Brain Computer Interfaces (BCI) testbed lab hosted by the University of Twente (UT) and carried out in cooperation with Thales. Ethical approval has been attained at the UT, and will be requested at the UU in the near future.  

\subsection{Methods}
\subsubsection{Electroencephalogram (EEG)}
The first modality is a technique  called electroencephalogram (EEG), which detects electrical activity in the brain using electrodes. Research by \cite{schirrmeister2017deep} contrasted several differently designed convolutional neural networks (ConvNets) against the baseline method for EEG data (FBCSP). Altogether, a deep ConvNet with four convolutional-max-pooling blocks was found to perform well, and will be utilized consequently \cite{schirrmeister2017deep}.  

\subsubsection{Galvanic Skin Response (GSR)}
The second modality is the technique called Galvanic Skin Response (GSR), measuring sweat gland on the hands and hereby inferring arousal. This technique was found to be an objective predictor of MWL \cite{shi2007galvanic}. For analyzing GSR data, a hybrid model combining a convolutional neural network (CNN) with a long-short-term memory (LSTM) model is proposed \cite{sun2019hybrid}. Such a model hybrid model was contrasted with a regular CNN (2 layers), after which the hybrid model was found to perform best \cite{dolmans2020perceived}. Consequently, a model stacking 2 convolutional layers with 2 LSTM layers will be utilized.  

\subsubsection{Photoplethysmography (PPG)}
The third modality constitutes Photoplethysmography (PPG), which is a technique utilized to measure heart rate, and is also proven to be a suitable indicator of MWL \cite{zhang2018evaluating}. A hybrid model, stacking two convolutional layers and two LSTM was found to perform best, for which this model will be utilized \cite{biswas2019cornet}.

\subsection{Model Evaluation}
The performance of the real-time framework will be assessed by contrasting it with a non-real-time framework. Specifically, the quality of performance for each modality individually will be assessed. This validation will be endeavored by means of six widely utilized performance measures within the fields of deep and machine learning: accuracy, sensitivity, specificity, PPV, NPV and F1. The framework that performs best across these measures will be considered the optimal framework. 

\newpage
\bibliographystyle{apacite}
\bibliography{References}
\end{document}
